\documentclass{frontiersSCNS}
\usepackage{url,lineno}
\linenumbers
\copyrightyear{}
\pubyear{}

\def\journal{Human Neuroscience}
\def\DOI{}
\def\articleType{Research Article}
\def\keyFont{\fontsize{8}{11}\helveticabold }

\def\firstAuthorLast{Brooks, and Cid de Garcia}

\def\Authors{Teon Brooks\,$^{1,*}$, and Daniela Cid de Garcia\,$^{2,*}$}

\def\Address{$^{1}$New York University, Department of Psychology, New York, NY, USA\\
$^{2}$Universidade Federal do Rio de Janeiro, Department of Anglo-Germanic Languages, Rio de Janeiro, RJ, BR}

\def\corrAuthor{Teon Brooks}
\def\corrAddress{Department of Psychology, New York University, 6 Washington Place, 2nd Floor, New York , NY, USA}
\def\corrEmail{teon@nyu.edu}

\extraAuth{Daniela Cid de Garcia\\Department of Anglo-Germanic Languages, Federal University of Rio de Janeiro, Av. Horácio Macedo, 2151, Sala D204, CEP 21941-917, Cidade Universitária, Rio de Janeiro - RJ, BR, cid.daniela@gmail.com}


\begin{document}
\onecolumn
\firstpage{1}

\title[Integrating morpheme form and meaning]{Evidence for Morphological Composition in Compound Words using MEG}
\author[\firstAuthorLast ]{\Authors}
\address{}
\correspondance{}

\topic{Morphologically complex words in the mind/brain}
\maketitle

\begin{abstract}

Psycholinguistic and M/EEG studies of lexical processing show convergent evidence for morpheme-based lexical access that involves early decomposition into constituent morphemes and later combinatorial semantic operations.  Considering that both semantically transparent (e.g. sailboat) and semantically opaque (e.g. bootleg) compounds undergo decomposition at the earlier stages of lexical processing, subsequent combinatorial operations should account for the difference in the meaning retrieval of these different word types.  In this study we use magnetoencephalography (MEG) to pinpoint the neural bases of this combinatorial stage in English compound word recognition and production. MEG data were acquired while participants performed a word naming task in which three word types (transparent, opaque, and simplex) were contrasted in two priming conditions (repetition and constituent). Analysis of onset latency reveal shorter latencies to speak for compound words than simplex words when primed, providing evidence of morphological decomposition. Analysis of associated brain activity have uncovered a region of interest implicated in morphological composition, the Left Anterior Temporal Lobe (LATL). Previous studies using sentences and phrases have highlighted the LATL as performing computations for basic combinatorial operations. Results are in tune with decomposition models and suggest the existence of a lexical processing dynamic that is modulated by morphological complexity for parsing structure and by semantic transparency for constructing meaning.

\tiny
 \keyFont{ \section{Keywords:} compounds, MEG, left anterior temporal lobe (LATL), word naming, morphology, semantic transparency, morphological decomposition, morphological composition } 
%All article types: you may provide up to 8 keywords; at least 5 are mandatory.
\end{abstract}

\section{Introduction}

	Some words are simple and some words are not. This, at first, sounds like a very trivial tautology, but the controversy over whether multi-morphemic words are simply stored in whole word form \citep*{Butterworth:1983, Giraudo:2001} or always constructed from their morphemic parts has been entertaining, provocative, and contentious in the field of lexical processing for the last forty years \citep*{Rastle:2003}. A comprehensive description of how words are both stored and retrieved requires the understanding of how form and meaning are connected, and how this connection unfolds in time in natural speech. Understanding the mechanisms involved in the comprehension of linguistic expression involves examining how perceptual information is analyzed from the moment in which the input is presented until the activation of the intended meaning happens.

	The potential contrast between simplicity and complexity in word storage and retrieval was first discussed in the classic affix-stripping model \citep*{Taft:1975}, which proposed that lexical access always proceeds via the stem of morphologically complex words.  Taft and Forster showed that pseudo-complex words with real stems took longer to reject in a lexical decision task (and were often selected incorrectly as words), than pseudo complex words in which a real prefix was followed by a non-existent stem. This was very early evidence that structure within a word matters, and is consulted online in particular to achieve lexical access. The evidence has mounted both in favor \citep*{Taft:2004, Marslen-Wilson:1994, Rastle:2004} and against full decomposition \citep*{Butterworth:1983, Giraudo:2001}, but the arguments against morphological sensitivity in lexical access have lost ground in the light of mounting evidence, giving rise to processing models where morphological decomposition is a necessary stage in processing. These psychological models make a variety of predictions as to the stages and time-course of lexical access, but there is a lack of evidence for the anchoring of the stages in particular areas of the brain. This study seeks to identify the area responsible for the composition of morpheme meanings. This area should be sensitive only to the composition within complex words whose morpheme meaning have a textit{transparent} semantic relationship to the overall meaning as compared to complex words whose morphemes do not share a semantic relationship, textit{opaque}.

	One way to look at lexical processing is to see if activating morphological structure can modulate the accessibility of a complex word. Some cross-modal priming studies \citep{Marslen-Wilson:1994} showed that priming in lexical decision between words that shared a stem only occurred when the prime and target had related meanings (e.g., departure primed depart but department did not) while other studies \citep*{Zwitserlood:1994} found that priming did not depend on a semantic relationship between the prime and target. However, masked priming studies that manipulate semantic transparency quite generally find facilitation effects regardless of whether prime and target share the same morphological root \citep*{Rastle:2004, Longtin:2003, Fiorentino:2007, McCormick:2008}.  Facilitation effects are found for every complex word that can be segmented in existing morphemes, which means that masked prime/unmasked target pairs like \textit{corner-corn} and \textit{bootleg-boot} speed recognition of the target facilitation as much as pairs like \textit{cleaner-clean} and \textit{teacup-tea}, even though there is no morphological relationship in the former pairs.

	If decomposition is performed for every word that can be exhaustively parsed into the forms of existing morphemes, regardless of whether the word is indeed morphologically complex, research on visual word recognition should shift the focus from decomposition to the subsequent mechanisms engaged to activate the actual meaning of a complex target word.  \citet*{Meunier:2007} suggest that word activation comes to play in stages, which include at least one early stage for morphological decomposition and a later stage for semantic integration of the morphological pieces. \citet*{Fiorentino:2013} presents evidence for a morpheme-based route for word activation that includes decomposition into morphological constituents and combinatorial processes operating on these representations.  Since previous studies have shown that early decomposition triggered by morphological structure happens automatically for transparent and opaque words, the difference between these two word types may manifest itself during the later stage of combinatorial operations.

	Another way to look at lexical processing is to look at composition. There are proposals for a general binding mechanism for basic composition proposed by \citet*{Bemis:2011} that may play a role at the word-level between words in a phrase. In a minimum composition paradigm, \citep{Bemis:2011} found that two composable items,in an adjective-noun phrase evoked more activation in the left anterior temporal lobe, LATL, than two non-composable items, a random letter string and word. This was taken as evidence that the most basic of combinatorial processing is supported by the LATL. This region should thus be also responsible for combinatorial operations with morphemes. Semantic transparency, the degree to which the combination of morpheme meanings corresponds to the overall word meaning, can be used to test whether the LATL is engaged when determining the semantic fit between morphemes within a complex word. In this experiment, we will use compounds to vary the semantic transparency of morphologically complex words since they consist of existing free morphemes with different levels of semantic relations between constituent and whole compound meanings. Thus, we expect semantically consistent word types such as transparent compounds (e.g. mailbox) to elicit activity in this region greater than simple words since their meanings derive from the composition of their morphemic parts, whereas we would not expect to see the same increase of activity for semantically opaque compounds (e.g. bootleg) since there is no relationship between its parts and its meaning. Since compounds are formed from words and have structures and meanings that resemble adjective-noun phrases, this provides an interesting basis to studying effects of intralexical semantic composition as an analogue to composition at the phrase level.

	Thus, a model of complex word recognition would require at least these two stages of process: parsing into basic units (decomposition), and the composition of these word forms into a complex meaning. To unpack these stages, we propose using two types of priming paradigms: a partial-repetition priming, similar to the ones used in masked priming studies, which will be used to investigate the decomposition effects on compounds, and a full-repetition priming, which will be used to investigate the composition effects on their morphemes. This study uses a word naming production task to investigate these stages involved in lexical processing. This task will be done while brain activity is recorded using MEG to investigate whether there is an area within the left temporal lobe that is responsible for morphological composition. This study contributes to the work of characterizing the neural bases of lexical processing of complex words by providing evidence for composition within compound words, while linking it to their neural correlates. Given the prior literature, we expect to find evidence of decomposition for compound words but not for simplex words. This would be a finding that fits in with the visual word recognition literature, specifically the masked priming literature, where there are facilitatory effects when priming morphologically complex words but not simplex words. However, we do not expect to find this overall benefit of morphological complexity in composition. Since composition is meaning governed, we only expect to find composition effects for transparent compounds in the brain activity. 

\section{Material \& Methods}

\textit{Participants.} Eighteen right-handed native speakers of English, with normal or corrected vision, participated in this experiment. Three participants were excluded due to large number of trial rejections caused by a noise interference (\textgreater 25 \%). Details for rejection are described in the procedure.

\textit{Material.}  All stimuli consisted of English bi-morphemic compound (e.g. teacup) and morphologically simplex (e.g. spinach) nouns, matched for length and surface frequency. We manipulated semantic transparency, including fully semantically transparent (e.g. teacup) words, in which both constituent morphemes have a semantic relationship to the meaning of the whole compound, and fully semantically opaque words (e.g. hogwash), in which neither of the constituent morphemes have a semantic relationship to the compound meaning.

	Three hundred eleven English compounds were compiled from previous studies \citep*{Drieghe:2010, Fiorentino:2007, Fiorentino:2009, Juhasz:2003} and categorized in terms of semantic transparency by means of a semantic relatedness task conducted using the Amazon Mechanical Turk tool. In this task, 20 participants were asked to judge, on a 1-7 scale, how much each constituent of the compounds related to the whole word.  On the scale, 1 corresponded to “unrelated” and 7 corresponded to “very related”.  Each participant was randomly presented with one of the constituents of each compound.  Compounds were classified as semantically opaque (henceforth opaque) if the sum of the scores of their constituents were within the interval 2-6, and as semantically transparent (henceforth transparent) if the sum were within the interval 10-14. For example, the opaque compound \textit{deadline} received a summed rating of 3.76 with \textit{dead} contributing a transparency rating of 1.44 and \textit{line} contributing a rating of 2.32. Similarly, the compound \textit{dollhouse} received a summed rating of 11.79 with \textit{doll} contributing a transparency rating of 6.47 and \textit{house} contributing a rating of 5.32. Sixty compounds were selected for each word type. This method of semantic transparency norming and the ratings were consistent with the mentioned prior studies. 
	The morphologically simple words (henceforth simplex: e.g. spinach) were pooled from \citet{Rastle:2004} and the English Lexicon Project \citep*{Balota:2007}.The simplex words were selected to have an embedded word within it but with no morphological or semantic relationship between the embedded word and the whole word. Also, these words were constrained and selected such that if the embedded word were removed, the remaining word part would have no meaning or use as a morpheme.
 
\textit{Design.}  The three different word types were contrasted in two priming conditions: full repetition and partial (constituent) repetition (See Table \ref{tab:Design-Matrix}).  For the repetition priming condition, the same compound was used as prime and target (e.g. TEACUP-teacup). For the constituent priming, we used the first constituent of the compound as the prime (e.g. TEA-teacup). For the simplex condition, the embedded word was used as the constituent in the constituent priming condition (e.g. SPIN-spinach). These two priming conditions were paired to control conditions in which the prime had no semantic relationship to the target (e.g. DOORBELL-teacup; DOOR-teacup). 

\begin{table}
\begin{tabular}{|c||c|c||c|c||c|c||}
\hline 
\multicolumn{1}{|c||}{\multirow{}{}{}} & \multicolumn{2}{c||}{Transparent} & \multicolumn{2}{c||}{Opaque} & \multicolumn{2}{c||}{Simplex}\tabularnewline
\cline{2-7} 
 & prime & target & prime & target & prime & target\tabularnewline
\hline 
\hline 
control & doorbell & teacup & heirloom & hogwash & brothel & spinach\tabularnewline
\hline 
repetition & teacup & teacup & hogwash & hogwash & spinach & spinach\tabularnewline
\hline 
\hline 
control & door & teacup & heir & hogwash & broth & spinach\tabularnewline
\hline 
partial-repetition & tea & teacup & hog & hogwash & spin & spinach\tabularnewline
\hline 
\end{tabular}\caption{\label{tab:Design-Matrix} Design Matrix}
\end{table}

\textit{Procedure.} All participants read all the items in all conditions (720 total), which were divided in three lists of 240 words and randomized within each list.  The order of presentation of the lists was counterbalanced between subjects.  The experimental task was word naming: subjects were presented with word pairs, and they were asked to read out loud the second word of each pair.  This task allowed for the use of every presented item to be included in the analysis. Each trial began with the presentation of a fixation cross, followed by the prime, then the target. Each of these visual presentations was presented for 300 ms with a 300ms blank. We recorded the onset latency to speech and the utterance from each subject for behavioral analysis.

Before the experiment, the head shape of each participant was digitized using the Polhemus Fastscan system, along with five head position indicator points, which are used to co-register the head position with respect to the MEG sensors during acquisition.  Electromagnets attached to these points are localized after the participants are lying within the MEG sensor array, allowing for coordination of head and sensor coordinate systems.  The head shape is used during the analysis to co-register the head to participants’ MRIs. For half of the participants, MRIs were not provided; therefore, we scaled the common reference brain that is provided in FreeSurfer to fit the size of these participants’ head.

	During the experiment, participants remain lying in a magnetically shielded room as their brain activity is monitored by the MEG gradiometers. The experimental items were projected onto a translucent screen so the participant could read and perform the task. The MEG data were collected using a axial whole-head gradiometer system with 157 channels and three reference channels (Kanazawa Institute of Technology, Nonoichi, Japan).  The recording was conducted in direct current (DC), that is, without a high-pass filter, and with a 300 Hz low-pass filter and a 60 Hz notch filter.

\textit{Analysis.} We examined onset latency, the reaction time to speak, to evaluate the effects of morphological decomposition based on \citep{Fiorentino:2007}. Since reaction time is sensitive to lexical processes, compounds should differ from single words, with processing associated with decomposition, composition, as well as properties of the constituents rather than the whole word. A non-decompositional account predicts no differences due to word structure, if the words are correctly matched for relevant whole word properties. Thus, onset latency can be used to disentangle whether or not there is a decomposition effect. The behavioral data were analyzed using traditional analysis of variance for the Word Type by Partial-Repetition priming interaction model.
Partial-repetition priming in lexical decision tasks has been used to demonstrate the accessibility of morphemes within complex words (Rastle et al. 2004). Similar behavioral effects have also been found using word naming (see Neely, 1991 for a comparative review of lexical decision and word naming). Therefore, the evidence of decomposition effects can be in observed in the reaction time to speak, \textit{onset latency}. Prior research leads to the prediction that we should observe a facilitative effect of shorter onset latency for the complex word types as compared to their simplex word counterparts. 


After brain data acquisition, we applied a Continuously Adjusted Least-Squares Method \citep{Adachi:2001}, a noise reduction procedure in the MEG160 software (Yokogawa Electric Corporation and Eagle Technology Corporation, Tokyo, Japan) that subtracts noise from the MEG gradiometers based on noise measurements at the reference channels positioned away from the head.  The data were bandpass filtered between 1-40 Hz using a IIR filter.  The recording of the whole experiment was segmented in epochs of interest, from -200 ms before to 600 ms after stimulus onset.  We rejected trials in which the maximal peak-to-peak amplitude exceeded the limit of 4000fT and we equalized the trials per condition for proper comparison. Sensor channels were marked as bad and discarded for each subject if the channel’s peak-to-peak rejection exceeded 10\%. 

A noise-covariance matrix was computed for each participant using an automated model selection procedure citep*{Engemann:2014} on the baseline epoch from -200 ms to the onset of the presentation of the fixation cross.  For participants with MRIs, cortical reconstructions were generated using FreeSurfer resulting in a source space of 5124 vertices (CorTechs Labs Inc., La Jolla, CA and MGH/HMS/MIT Athinoula A. Martinos Center for Biomedical Imaging, Charleston, MA). For participants without MRIs, the headshape-constrained FreeSurfer average brain was used. A boundary-element model (BEM) method was used to model activity at each vertex to calculate a forward solution. An inverse solution was generated using this forward model and noise-covariance matrix, and was computed with a fixed-orientation constraint requiring dipole sources to be normal to the cortical surface.
The sensor data for each subject was then projected into their individual source space using a cortically-constrained minimum norm estimate (all analyses were conducted using MNE-Python: \citet*{Gramfort:2013, Gramfort:2013a}) resulting in noise-normalized dynamic statistical parameter maps (dSPMs).

This analysis was conducted with the objective of isolating the activity related to basic computational mechanisms involved in the composition of previously-activated constituent morphemes in the compounds. In order to examine this mechanism, we considered the neurophysiological data related solely to the silent reading of the primes in the repetition condition. The priming results also served to make sure that these words were actually read by the participants, because they affected the activation of the target.  The composition effect was examined as if a list of isolated words were read silently by the participants, as the words used as primes were assumed not influenced by their preceding trial. In this respect, the method of analysis is analogue to that adopted by \citet{Zweig:2009}, in which the authors directly compare complex (derived) with mono-morphemic words, thus aiming to find decomposition effects that are not dependent on priming.  

For this analysis, our design (Table \ref{tab:Primes-Analysis}) reduces to the simple comparison between compounds (e.g. TEACUP) and simplex words (e.g. SPINACH) of the same size that served as primes in the repetition condition (e.g. TEACUP-teacup) described above in the Design section. Since, for this analysis, we use neurophysiological data related to the silent reading of the words that served as primes, there is no behavioral data for these words.  By these means we also avoid artifacts associated with voluntary movements that can compromise the analysis of the effects of interest to the study \citep*{Hansen:2010}.
 
We examined the neural activity localized in the entire left temporal lobe. This region was selected based on composition effects found with sentences \citep{Friederici:2000} or adjectival phrases \citep{Bemis:2011}.
In order to verify if there was increased activity for compounds in this area, a t-test was performed on the difference of a compound word type (opaque, transparent) and its simplex control word from 100 ms to 600 ms after the stimulus onset. The p-value map of the brain was generated for the time series and spatiotemporal clusters were identified for contiguous space-time clusters that had a p-value of less than .05 and a duration of at least 10 ms. The t-values from the clusters were summed for those clusters that met these criteria. Then, a non-parametric, cluster-based permutation test was performed on the summed t-values within the left temporal lobe on these clusters. A distribution was computed from calculating the summed t-values from clusters formed after shuffling the condition labels. The corrected p-value was determined from the percentage of clusters that were larger than the original computed cluster \citep{Maris:2007}. These tests were computed using the statistical analysis package for MEG data, Eelbrain, (https://pythonhosted.org/eelbrain/).

\begin{table}
\centering{}%
\begin{tabular}{|c|c|}
\hline
Word Types & Examples\tabularnewline
\hline
\hline
Opaque & hogwash\tabularnewline
\hline
Transparent & teacup\tabularnewline
\hline
Simplex (control) & brothel\tabularnewline
\hline
\end{tabular}\caption{\label{tab:Primes-Analysis} Primes Analysis}
\end{table}

\section{Results}
 
\textbf{Morphological decomposition.} Behaviorally, we found a significant effect of  constituent priming [F(1,17) = 25.91, p \textless .001], but most critically an interaction of word type by constituent priming [F(2,17) = 9.24, p \textless .001] (Figure \ref{fig:latency}). This effect shows that there is a greater facilitation in word naming for compound words than for morphologically simple words when primed. This results show that even in word production, there is sensitivity to morphological structure above and beyond orthographic and phonological overlap (Table \ref{tab:latency}). 

\begin{table}
\begin{center}
\begin{tabular}{lrrrrr}
\toprule
& \multicolumn{1}{c}{\textbf{SS}} & \multicolumn{1}{c}{\textbf{df}} & \multicolumn{1}{c}{\textbf{MS}} & \multicolumn{1}{c}{\textbf{F}} & \multicolumn{1}{c}{\textbf{p}} \\
\midrule
Word Type & 2465.48 & 2 & 1232.74 & 1.82 & .178 \\
Partial-Repetition Priming & 17906.72 & 1 & 17906.72 & $25.91^{***}$ & $< .001$ \\
Word Type x Partial-Repetition Priming & 3578.01 & 2 & 1789.01 & $9.24^{***}$ & $< .001$ \\
\midrule
Total & 910981.34 & 107 \\
\bottomrule
\end{tabular}\caption{\label{tab:latency} Onset Latency Time Differences}
\end{center}
\end{table}

\textbf{Morphological composition.} Results reveal significant effects of lexical complexity identified by the permutation tests on the difference of compounds to their simplex counterparts  within the temporal lobe. There were two significant clusters associated with the difference of transparent compounds and simplex words. The first cluster was  localized toward the anterior part of the left temporal lobe from 250 to 470ms ($\sum$ t = 4552.3, p \textless .05, Figure \ref{fig:transparent_primes}). The second cluster of activity was localized to the posterior part of the superior temporal lobe from 430 to 600 ms ($\sum$ t = 5654, p \textless .05, Figure \ref{fig:transparent_2_primes}). However, there were no significant clusters found for the difference of opaque compounds and simplex words within the temporal lobe.  
Taken together, our results that the anterior part of the left anterior lobe is implicated in this difference found between morphologically complex words and their simple counterparts. This increase in activity for morphologically complex words occurs for semantically transparent compounds but not for semantically opaque ones, that is, this engagement occurs for complex words whose morphemes have a semantic relationship to the overall meaning of the word. This same pattern of activity seems to continue as it subsequently engages in the posterior part superior temporal lobe.  

\section{Discussion}
	Analyses of the different word types in isolation reveal very consistent evidence that there is a difference in how simplex and complex words are processed in the brain. The behavioral results revealed there is a stage in lexical access that is sensitive to the morphological complexity of words. The onset latency interaction effect where complex words were produced faster than simple words when primed by their constituent morpheme is largely consistent with the results within the masked priming literature on word recognition, and gives further evidence that where there is a there is a decomposition stage in lexical access where complex words are parsed into their morphemes \citep{Rastle:2004, Taft:2004, Morris:2007, McCormick:2008, Fiorentino:2009}. The parsing operation occurs independent of semantic relationship between constituent morphemes and their complex word. 
	Since early activation of constituents via morphological decomposition happens irrespective of semantic transparency, what differentiates transparent and opaque compound must happen, thus, during a later stage of morphemic composition.
The increased activity found for transparent compounds in anterior temporal lobe from 250 to 470ms provides evidence with stage in lexical access where meanings of the morpheme play a part in access the overall meaning of the word. \cite{Bemis:2011} show combinatorial effects on the LATL for adjectival words at around 250 ms after the critical word is presented.  The difference in timing could be explained by the different time points in which we time lock the onset of the stimulus. In \cite{Bemis:2011}, the onset coincides with the onset of the noun \textit{boat} in the phrase \textit{red boat}, whereas in our study the critical stimulus is the entire compound textit{sailboat}.

The increased activation in the posterior temporal lobe for transparent compounds from 430 to 600 ms that follows the activity in the LATL is consistent with the fact that this region is involved in lexical retrieval \citep{Lau:2008, Hickok:2007}. \cite{Lau:2008} proposed that the posterior region of the temporal lobe as the best candidate for the lexical storage of words. Since the LATL is responsible for composing the meaning of the constituent morphemes, the posterior temporal lobe would be responsible for retrieval information from its stored lexico-semantic representation. This region is also engaged sound-to-meaning transformation \citep{Binder:2000}, which would include the retrieval of phonological information. 
This study is in tune with decomposition models from visual word recognition literature and provides the neural basis for a stage in lexical access involved in the composition of meaning within compound words, thus helping to disentangle cognitive processes that are compressed in response times alone.  Bridging results from psycholinguistic research with MEG recordings of brain activity, the emerging results suggest that the recognition of compounds is achieved at distinct stages that are governed by semantics. We showed that the course of activation varies in terms of word complexity and semantic transparency. 

\section*{Disclosure/Conflict-of-Interest Statement}
The authors declare that the research was conducted in the absence of any commercial or financial relationships that could be construed as a potential conflict of interest.

\section*{Author Contributions}
Authors Teon L. Brooks and Daniela Cid de Garcia share first-authorship as they have both equally contributed to the paper.

\section*{Acknowledgement}
We would like to thank Jeff Walker of the NYU MEG Lab for his help while running participants.

\paragraph{Funding\textcolon} 		 	 	 		
					
This work is supported by the National Science Foundation under Grant No. BCS-0843969, and by the NYU Abu Dhabi Research Council under Grant No. G1001 from the NYUAD Institute, New York University Abu Dhabi.  The work of Teon Brooks was supported by the National Science Foundation Graduate Research Fellowship under DGE-1342536. The work of Daniela Cid de Garcia was supported by the Coordination for the Improvement of Higher Education Personnel and the Fulbright Commission under the Mutual Educational Exchange Act, sponsored by The United States of
America Department of State, Bureau of Educational and Cultural Affairs.

\bibliographystyle{frontiersinSCNSENG} % for Science and Engineering articles
\bibliography{library}

\section*{Figures}

\begin{figure}
\begin{centering}
\includegraphics[scale=0.75]{images/latency_constituent_analysis}
\par\end{centering}
\caption{\label{fig:latency} Constituent Priming Differences in Latency Times}
\end{figure}

\begin{figure}
\begin{centering}
\includegraphics[scale=0.50]{images/transparent_prime_brain_analysis}\includegraphics[scale=0.50]{images/transparent_prime_analysis}
\par\end{centering}
\caption{\label{fig:transparent_primes} Transparent vs Simplex Difference in LATL}
\end{figure}

\begin{figure}
\begin{centering}
\includegraphics[scale=0.50]{images/transparent_2_prime_brain_analysis}\includegraphics[scale=0.50]{images/transparent_2_prime_analysis}
\par\end{centering}
\caption{\label{fig:transparent_2_primes} Transparent vs Simplex Difference in posterior Temporal Lobe}
\end{figure}


\end{document}