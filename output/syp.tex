%% LyX 2.0.2 created this file.  For more info, see http://www.lyx.org/.
%% Do not edit unless you really know what you are doing.
\documentclass[12pt,english]{article}
\usepackage[T1]{fontenc}
\usepackage[latin9]{inputenc}
\usepackage{geometry}
\geometry{verbose,tmargin=1in,bmargin=1in,lmargin=1in,rmargin=1in}
\usepackage{array}
\usepackage{multirow}
\usepackage{setspace}
\usepackage[authoryear]{natbib}
\doublespacing

\makeatletter

%%%%%%%%%%%%%%%%%%%%%%%%%%%%%% LyX specific LaTeX commands.
%% Because html converters don't know tabularnewline
\providecommand{\tabularnewline}{\\}

%%%%%%%%%%%%%%%%%%%%%%%%%%%%%% User specified LaTeX commands.
\usepackage{booktabs}

\makeatother

\usepackage{babel}
\begin{document}

\title{Title}


\author{Teon Brooks}
\maketitle
\begin{abstract}
Words tend to shorten when they are repeated. They are also pronounce
with shorter delay when they have been previously mentioned. It is
unclear whether repeating part of a word This study reports that 
\end{abstract}
\listoftables
\tableofcontents{}


\section{Introduction}

Variation in speech production is informative to how mental representations
of lexical items are constructed and maintained. These representations
stand at the intersection of the mental lexicon and working memory.
\citet{Levelt:1999uq} proposed a multi-stage model of lexical access
for speech production where we begin with the conceptualization of
the lexical item, which leads to its retrieval, and then ultimately
its articulation. This basis of the \citet{Levelt:1999uq} model works
the independence of the conceptual system and the articulatory motor
system. These systems are interfaced with their respective encoding
stage. The conceptual system has a morphophonological encoding stage
that is responsible for giving form to the concept while the articulatory
motor system is fed input from phonetic encoding stage where articulatory
gestures are planned for speech. The crux of this model lies at a
separation of meaning from its realized form. In order for this to
be achieved, it is critical to understand the nature of these encoding
processes that are involved in transforming concepts into an fully
articulated form.


\subsection{Encoding}

Encoding encompasses a set of operations necessary for the transmission
of the conceptual message. There are various models offering predictions
of the nature of this encoding ranging from phonological only encoding
to those with a mediating morphological stage. \citet{Meyer:1990fk,Meyer:1991uq}
found that repetition of phoneme clusters alone can lead to preparatory
effect in production with results suggesting that syllables are planning
units for speech. In contrast, \citet{Roelofs:2002fk} reported that
production of polymorphemic words relied on the use of their morphemic
constituents for the encoding process. These word forms are not only
used in production but they are accessible regardless of word's semantic
transparency, or the well-formedness of word forms' composition to
the word's overall meaning. \citet{Levelt:1999uq} proposes that encoding
follows from lexical selection, which is non-decompositional by its
very nature. This model predicts that both morphological access and
encoding operates at a post-lexical stage, not affecting lexical retrieval.
This encoding operation specifies word forms as planning units. However,
this model does not specify whether the word forms are accessible
in words with opaque, non-compositional, structure. Work by Dohmes
suggest that there is a form representational level in speech production
where morphologically complex words are represented by their constituent
morphemes independent of semantic transparency.


\subsection{Priming}

But word durations, and the shortening of frequent words in particular,
have proven a highly sensitive diagnostic for frequency effects in
language processing (Bell et al. 2003, Bybee 2001, Fosler-Lussier
\& Morgan 1999, Jurafsky 2003, Jurafsky et al. 2001b, Krug 1998, Losie-
wicz 1995, Pluymaekers et al. 2005a,b).\citet{Gahl:2008nm} showed
that word frequency can modulate the word duration of homophones.
Repetition priming is a phenonemon that is generally agreed upon and
have good coverage in its description of what happens when information
is repeated. \textquoteleft{}With repetition, neuromotor routines
become more compressed and more reduced\textquoteright{} (2001:78). 

Therefore, if word durations reflect only effects of form frequency,
never of lemma frequency, then this should cast doubt on the lemma
level as a locus of frequency information.

Why does word duration vary? Several mutually compatible explanations
have sug- gested themselves: word durations tend to shorten, and articulatory
effort tends to be reduced, as a function of repetition within a discourse
(Bard et al. 2000, Fowler 1988, Fowler \& Housum 1987, Fowler et al.
1997, Shields \& Balota 1991), predictability within an utterance
(Gregory et al. 1999, Hunnicutt 1985, Jurafsky et al. 2001a,b, Lieberman
1963), and neighborhood density (Wright 2004).

In all of these cases, high-probability forms tend to reduce, and
low-probability forms tend to lengthen or otherwise be hyperarticulated.

For example, Newmeyer argues: \textquoteleft{}It is a truism that
the more often we do something, the faster we are able to do it. That
is as true for language as for anything else\textquoteright{} (2006:401).
Priming is a technique that can be used to probe at the representation
status of a word previously activated \citet{Arnold:2012fk} used
a priming paradigm where found that when priming a word in a discourse
modulates its production resulting in reduction in their duration. 

\citet{Balota:1989kx} explored the effects of priming of related
items parametrically using various SOAs. Using associative priming,
they found that a related response cue led to a reduction in the target
word's production duration. Interestingly, they found that SOA of
the associated cue to be independent of the production duration of
its target. This model predicts that there is some semantic governance
for word production where only semantically associated items lead
to facilitatory effects. 

The study of these effects on the encoding operation in word production
has largely remained uncharted territory. Few studies have looked
at the mechanism of how the internal structure of word may lead to
its production. \citet{Roelofs:2002fk} have found a facilitatory
priming effect on latency times for both transparent and opaque compounds
but no benefit for mono-morphemic words. They argued that word forms
(morphemes) may be used as planning units in production, regardless
of its semantic transparency. This morphologically governed production
model seems to parallel finding in lexical decision studies using
masked priming where there is a benefit to internal structure in determining
the lexical status of word regardless of its transparency \citep{Rastle:2004bh,McCormick:2008mn}.
These effects highlight the fact that these benefits are not merely
form overlap but determined by morphological structure. However, it
is unclear whether this applies for words with pseudo-morphological
structure, that is, is there access to the constituent word forms
in semantically opaque words? Since word forms are predicted to be
planning units in speech, if opaque words are parsed into their word
forms (decomposition), this operation would need to be semantically
blind. In this study, we use compound words since they provide a rich
testing bed for assessing these predictions. Spatially unified bi-morphemic
compounds were used in this study. Compounds can vary in the degree
of semantic fit of its constituents to its overall meaning, often
referred to as semantic transparency. The constituents of a fully
semantically transparent compound both contribute to the word's overall
meaning, whereas neither constituent of a fully semantically opaque
compound contribute to its meaning. With compound's unique property
of having only words as their constituents, we can test the accessibility
of the constituents within a priming paradigm. 

A morphological based production model would predict that there would
a reduction in the duration for all internally structured words regardless
of their transparency giving a contrast between the mono-morphemic
words and the compound words. In a semantics-governed production model,
we expect their to be facilitatory effects for internally structured
words with transparent meaning. This would mean there would be facilitatory
effects for the transparent but no effects for the opaque compounds
and mono-morphemics. Moreover, opaque compounds' internal structure
would be merely remnant traces making them effectively lexicalized
mono-morphemics. This study argues for a semantics-free parsing stage
as predicted by \citet{Roelofs:2002fk} where word forms are used
as planning units for production. 





 




\section{Methods}


\subsection{Stimuli}

A sizable corpus of compounds were compiled from prior studies (\citet{Drieghe:2010vn,Fiorentino:2007kw,Fiorentino:2009:xi,Juhasz:2003ka})
and were normed using a semantic relatedness survey administered through
Amazon Mechanical Turk. The semantic relatedness survey asked participants
to rate on a Likert scale from 1 to 7 how related the meaning of the
one of compounds' constituents is to the compound itself where ``1''
was not related and ``7'' was very related. 20 participants were
randomly given only one constituent per compound to judge. The ratings
were largely consistent with prior work. 

We defined constituents as semantically opaque (e.g. deadline) with
a score between 1 to 3 and semantically transparent (e.g. dollhouse)
between 5 to 7. Compounds were considered fully opaque if their summed
ratings were 1-6 and fully transparent if 10-14. For example, the
opaque compound 'deadline' received a summed rating of 3.76 with 'dead'
contributing a transparency rating of 1.44 and 'line' contributing
a rating of 2.32. Similarly, the compound 'dollhouse' received a summed
rating of 11.79 with 'doll' contributing a transparency rating of
6.47 and 'house' contributing a rating of 5.32. Sixty fully opaque
and sixty fully transparent compounds were randomly selected from
the corpus. 

In addition to the opaque and transparent compounds, we included novel
compounds to look at production effects for words with possible morphological
structure but with no established overall meaning. Sixty novel compounds
were randomly generated using the remaining compounds from the relatedness
survey. These novel compounds were generated by concatenating randomly
selected constituents from their remaining respective pools. Since
these constituents are words that are likely to be compounded with
others, they make good candidates for novel word formation. To prevent
the random formation of an existing word, the word frequency of these
compounds were checked using the English Lexicon Project's word frequencies
\citep{Balota:2007fk}.

Lastly, to test whether duration reduction was merely a function of
phonological overlap, mono-morphemic words consisting of an embedded
but no possible internal morphological construction (e.g. BROTHel),
which we will simply referred to as mono-morphemics were added for
comparison. The mono-morphemic words were pooled from \citet{Rastle:2004bh}
and the English Lexicon Project\citep{Balota:2007fk}, were selected
to have overlapping punctuation between the mono-morphemic and its
embedded word (e.g. HATCH-hatchet vs. CORD-cordial). Sixty mono-morphemics
were selected for this study: thirty of them having the embedded words
at the beginning of the word (e.g. HATCH-hatchet) while thirty others
with the embedded words at the end (LOG-dialog) . 


\subsection{Participant Recruitment}

Participants were recruited from the New York City community. There
was a total of 17 participants.


\subsection{Experimental Design and Task}

This study contrasted four different word types, each with 60 items:
mono-morphemics, and transparent, opaque, and novel compounds. These
word types were compared in two types of priming, repetition and constituent.
For the repetition priming condition, the word of interest was shown
as both the prime and the target as compared with the constituent
priming where the constituent of word (for this study, first-constituent
only) is used as a prime for its whole word target. These priming
conditions were compared to their semantically unrelated controls.
Each word appeared once per block in one of the four different conditions
across four blocks. They were counterbalanced so that all words appeared
in all conditions per subject and were latin-squared for condition
order across subjects producing a completely within-subjects, fully
factorial design: Word Type (4) $\times$ Constituency (2) $\times$
Priming (2).

\begin{table}
\begin{tabular}{|c||c|c||c|c||c|c||c|c|}
\hline 
\multicolumn{1}{|c|}{\multirow{2}{*}{}} & \multicolumn{2}{c||}{Transparent} & \multicolumn{2}{c||}{Opaque} & \multicolumn{2}{c||}{Novel} & \multicolumn{2}{c|}{Simple}\tabularnewline
\cline{2-9} 
 & prime & target & prime & target & prime & target & prime & target\tabularnewline
\hline 
\hline 
control & doorbell & teacup & heirloom & hogwash & keybook & ladyfork & mailbox & spinach\tabularnewline
\hline 
identity & teacup & teacup & hogwash & hogwash & ladyfork & ladyfork & spinach & spinach\tabularnewline
\hline 
\hline 
control & door & teacup & heir & hogwash & key & ladyfork & mail & spinach\tabularnewline
\hline 
constituent & tea & teacup & hog & hogwash & lady & ladyfork & spin & spinach\tabularnewline
\hline 
\end{tabular}

\caption{Design Matrix}
\end{table}


The study used a repetition priming paradigm with word naming task.
Psychtoolbox \citet{Brainard:1997vn} was used for the presentation
of stimulus. In this task, the trial begins with a fixation cross,
which appears in the center of the screen, followed by a prime then
a target. Prime words are presented in all capital letters while target
words are presented in lowercase letters. All of the visual presentations
are for 300 ms with an inter-stimulus interval of 600ms. The objective
of the participant is to read aloud the target word in each trial.
The next trial does not proceed until a response is uttered. The onset
latency was measured from the presentation of the target word to the
threshold activation of the voice trigger. Their production was recorded
from the point at which voice trigger was activated. Each participant
was presented with 960 trials: 60 words in each of four word types
counterbalanced for each of the four condition.


\subsection{Analysis}

In this study, the effects of priming on both naming latency and word
durations were analyzed using a repeated-measures ANOVA. Naming latencies
were defined as the duration from the visual onset of the target stimulus
to the onset of speech. 

To obtain the duration measures from the production data, we use the
Penn Forced Aligner \citep{Yuan:2008kx}. We provided transcripts
for each target word where each constituent was considered a separate
word. A textgrid was created for each target word with boundary markings
for both the phone and word, which were generated from the HTK model.
The audio was checked for accuracy of the target word production.
These textgrids were hand-corrected with the following requirements:
We relied on the HTK model to determine the onsets of compound since
the detection of onset of stop burst have uncertainty, which would
not allow for a clear user criterion to be set. The offset of the
constituent boundaries were determined to be where there was no energy
in the spectrogram and waveform. \citet{Morrill:2012fk} These criteria
were allow uniformity with proper discretion from the author.


\subsection{Results}

\begin{table}[h]
\begin{center}
\begin{tabular}{lrrrrr}
\toprule
 & \multicolumn{1}{c}{\textbf{SS}} & \multicolumn{1}{c}{\textbf{df}} & \multicolumn{1}{c}{\textbf{MS}} & \multicolumn{1}{c}{\textbf{F}} & \multicolumn{1}{c}{\textbf{p}} \\
\midrule
wordtype & 6451.20 & 3 & 2150.40 & $6.81^{***}$ & $< .001$ \\
condition & 110985.57 & 1 & 110985.57 & $77.24^{***}$ & $< .001$ \\
wordtype x condition & 1021.75 & 3 & 340.58 & $0.93^{   \ \ \ }$ & .431 \\
\midrule
Total & 1345197.48 & 143 \\
\bottomrule
\end{tabular}
\end{center}\caption{\label{table:latency-identity}ANOVA table for Identity Priming Latencies}
\end{table}


\begin{table}[h]
\begin{center}
\begin{tabular}{lrrrrr}
\toprule
 & \multicolumn{1}{c}{\textbf{SS}} & \multicolumn{1}{c}{\textbf{df}} & \multicolumn{1}{c}{\textbf{MS}} & \multicolumn{1}{c}{\textbf{F}} & \multicolumn{1}{c}{\textbf{p}} \\
\midrule
wordtype & 5320.74 & 3 & 1773.58 & $3.42^{*  \ \ }$ & .024 \\
condition & 29958.08 & 1 & 29958.08 & $40.06^{***}$ & $< .001$ \\
wordtype x condition & 4612.17 & 3 & 1537.39 & $6.93^{***}$ & $< .001$ \\
\midrule
Total & 1208222.91 & 143 \\
\bottomrule
\end{tabular}
\end{center}

\caption{\label{table: latency-constituent}ANOVA table for Constituent Priming
Latencies}
\end{table}


\begin{table}[h]
\begin{center}
\begin{tabular}{lrrrrr}
\toprule
 & \multicolumn{1}{c}{\textbf{SS}} & \multicolumn{1}{c}{\textbf{df}} & \multicolumn{1}{c}{\textbf{MS}} & \multicolumn{1}{c}{\textbf{F}} & \multicolumn{1}{c}{\textbf{p}} \\
\midrule
wordtype & 5918.04 & 3 & 1972.68 & $7.74^{***}$ & $< .001$ \\
condition & 608.26 & 1 & 608.26 & $2.72^{   \ \ \ }$ & .128 \\
wordtype x condition & 459.79 & 3 & 153.26 & $0.48^{   \ \ \ }$ & .697 \\
\midrule
Total & 126517.05 & 95 \\
\bottomrule
\end{tabular}
\end{center}\caption{\label{table:duration-identity}ANOVA table for Identity Priming Duration}
\end{table}


\begin{table}[h]
\begin{center}
\begin{tabular}{lrrrrr}
\toprule
 & \multicolumn{1}{c}{\textbf{SS}} & \multicolumn{1}{c}{\textbf{df}} & \multicolumn{1}{c}{\textbf{MS}} & \multicolumn{1}{c}{\textbf{F}} & \multicolumn{1}{c}{\textbf{p}} \\
\midrule
wordtype & 12933.39 & 3 & 4311.13 & $18.91^{***}$ & $< .001$ \\
condition & 187.23 & 1 & 187.23 & $0.75^{   \ \ \ }$ & .405 \\
wordtype x condition & 2060.54 & 3 & 686.85 & $2.13^{   \ \ \ }$ & .116 \\
\midrule
Total & 122991.88 & 95 \\
\bottomrule
\end{tabular}
\end{center}\caption{\label{table:duration-constituent}ANOVA table for Constituent Priming
Duration}
\end{table}


\pagebreak{}\bibliographystyle{apalike2}
\addcontentsline{toc}{section}{\refname}\bibliography{/Users/teon/Dropbox/articles/library}

\end{document}
